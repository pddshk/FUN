% !TeX spellcheck = ru_RU
% !TeX TXS-program:compile = txs:///xelatex/[--shell-escape]

\documentclass[12pt]{extarticle}
\usepackage[no-math]{fontspec}
\usepackage[english,russian]{babel}
\usepackage{indentfirst}
\usepackage{xcolor}
\usepackage{tikz-cd}
\usepackage{amsmath}
\usepackage{amssymb}
\usepackage{xpatch}
\usepackage{hyperref}
\usepackage{amsthm}
\usepackage{mathrsfs}
\usepackage{yfonts}
\usepackage{titlesec}
\usepackage{unicode-math}
\usepackage[a4paper, top=1.5cm, bottom=1.5cm, right=1.5cm, left=1.5cm]{geometry}

\usepackage{xunicode}
\usetikzlibrary{babel}

% Font sepcifications
\setmainfont{PTSerif}[
    Extension = .ttf,
    UprightFont = *-Regular,
    BoldFont = *-Bold,
    ItalicFont = *-Italic,
    BoldItalicFont = *-BoldItalic
]

\setmathfont[bold-style=ISO]{latinmodern-math.otf}
\setmathfont[range={\varnothing}]{Asana-Math.otf}
\setmathfont[range=bb]{texgyrepagella-math.otf}

\AtBeginDocument{
    \renewcommand{\setminus}{\mathbin{\big\backslash}}%
}

%%%%%%%%%%%%%%%%%%%%%%%%%%%%%%%%%%%%%%%%%%%%%%%%%%%%%%%%%%%

%\renewcommand{\baselinestretch}{1.2857142857142858094476878250134177505970001220703125}\large\rm

\makeatletter

\xpatchcmd{\proof}{\@addpunct{.}}{}{}{}

\makeatother

\theoremstyle{definition}
\newtheorem{theorem}{\indent Теорема}[section]
\newtheorem{corollary}{\indent Следствие}%[section]
\newtheorem{lemma}{\indent Лемма}[section]
\newtheorem*{statement}{\indent Утверждение}
\newtheorem{definition}{\indent Определение}[section]
\newtheorem*{example}{\indent Пример}%[section]
\newtheorem*{remark}{\indent Замечание}
\newtheorem*{question}{\indent Вопрос}
\newtheorem*{excercise}{\indent Упражнение}
\newtheorem*{corollary*}{\indent Следствие}

\renewcommand{\qedsymbol}{$\lhd$}

\addto\captionsrussian{\renewcommand{\proofname}{$\rhd$}}

%%%%%%%%%%%%%%%%%%%%%%%%%%%%%%%%%%%%%%%%%%%%%%%%%%%%%%%%%%%%%%%%%
\newcommand{\pref}[1]{(\ref{#1})}

\newcommand{\factor}[2]{\raisebox{.35em}{$#1$}\kern-.27em\Big/\kern-.27em\raisebox{-.35em}{$#2$}}


\newcommand{\brac}[1]{\left(#1\right)}
\newcommand{\fbrac}[1]{\left\{#1\right\}}
\newcommand{\sbrac}[1]{\left[#1\right]}
\newcommand{\abrac}[1]{\langle#1\rangle}

\newcommand{\eps}{\varepsilon}

\newcommand{\RR}{\mathbb{R}}
\newcommand{\NN}{\mathbb{N}}
\newcommand{\ZZ}{\mathbb{Z}}
\newcommand{\bb}[1]{\mathbb{#1}}
\newcommand{\eqby}[1]{\stackrel{\mathrm{#1}}{=}}
\newcommand{\implby}[1]{\stackrel{\mathrm{#1}}{\implies}}
\newcommand{\eqdef}{\stackrel{\mathrm{def}}{=}}
\newcommand{\ddt}{\frac{d}{dt}}
\newcommand{\iffdef}{\stackrel{\mathrm{def}}{\iff}}
\newcommand{\dd}[2]{\frac{d #1}{d #2}}

\newcommand{\at}[2]{\left.#1\right\vert_{#2}}

\newcommand{\partiald}[2]{\frac{\partial #1}{\partial #2}}
\newcommand{\partialdd}[3]{\frac{\partial^2 #1}{\partial #2 \partial #3}}

\DeclareMathOperator{\Mat}{Mat}
\DeclareMathOperator{\pr}{pr}
\DeclareMathOperator{\id}{id}
\DeclareMathOperator*{\olim}{\overline{\lim}}
\DeclareMathOperator{\transposed}{^{T}}
\newcommand{\ollim}[1]{\olim\limits_{#1}}

%%%
\titleformat{\section}{\centering\Large\bfseries}{\S\thesection}{3mm}{}
\titlespacing{\section}{0cm}{1.5cm}{0.5cm}

%%%%%%%%%%%%%%%%%%%%%%%%%%%%%%%%%%%%%%%%%%%%%%%%%%%%%%%%%%%%
\begin{document}
    \begin{titlepage}
        \null
        \vfill
        \centering\Huge\bfseries Конспект лекций\\по функциональному анализу
        \vfill
        \null
    \end{titlepage}
    \setcounter{page}{2}
    \tableofcontents
    \newpage
    \section[Алгебры, идеалы, условия обратимости.]{Алгебра, идеал, максимальный идеал.\\Условия обратимости.}
    Рассмотрим векторное пространство $A$ над полем $F$ (или, вообще говоря, модуль $A$ над кольцом $R$).
    \begin{definition}[Алгебра]\label{def:Algebra}
        Пусть $R$~--- коммутативное кольцо с единицей, $A$~--- модуль над этим кольцом ($R$-модуль). Снабдим $A$ билинейным ассоциативным отображением $f\colon A\times A\to A$ и назовём это отображение умножением. Тогда $(A, +, \cdot)$ является кольцом.

        Полученная структура называется \textbf{алгеброй над кольцом $\symbf R$}.
    \end{definition}
    \begin{remark}
        В определении алгебры требовалась ассоциативность умножения. Вообще говоря это необязательно, однако в данном конспекте будут рассматриваться только ассоциативные алгебры, поэтому для краткости будем считать умножение ассоциативным.

        Также далее будут рассматриваться в основном векторные пространства над полями (частный случай модуля над кольцом).
    \end{remark}
    \begin{theorem}\label{th:Constructing algebra}
        Любое векторное пространство $V$ над полем $F$ можно доопределить до алгебры
    \end{theorem}
    \begin{proof}
        Доказательство проведём для конечномерного случая. Для бесконечномерного аналогично.

        Выберем некоторый базис векторного пространства $\{e_1,e_2,\dots,e_n\}$ и произвольные $n^2$ векторов $v_{11}, v_{12}, \dots, v_{nn}$. Определим умножение следующим образом:
        \begin{gather*}
            e_i\cdot e_j = v_{ij}\\
            \forall \alpha,\beta\in F, u,v\in V\\
            \alpha(\beta u) = (\alpha\beta)u\\
            \alpha(u + v)=\alpha u + \alpha v\\
            u(\alpha+\beta)=u\alpha+u\beta
        \end{gather*}

        Тогда в силу определения базиса имеем для любых векторов $a,b\in V$
        \begin{gather*}
            a=a_1e_1+a_2e_2+\dots+a_ne_n\\
            b=b_1e_1+b_2e_2+\dots+b_ne_n\\
            a\cdot b= (a_1e_1+a_2e_2+\dots+a_ne_n)\cdot(b_1e_1+b_2e_2+\dots+b_ne_n)=\\
            =a_1b_1v_{11}+a_1b_2v_{12}+\dots+a_nb_nv_{nn}
        \end{gather*}
    \end{proof}
    \begin{corollary*}
        Все дополнительные свойства (коммутативность, антикоммутативность и т.д.), которыми умножение обладает на базисных векторах, продолжаются на всё пространство.
    \end{corollary*}
    \begin{example}
        $V = \RR^3$, базис $\symbf{\vec i,\vec j,\vec k}$, умножение базисных векторов:
        \begin{center}
            \begin{tabular}{c|c|c|c}
                $\times$ & $\vec{\symbf i}$ & $\vec{\symbf j}$ & $\vec{\symbf k}$\\
                \hline
                &&&\\[-1em]
                $\vec{\symbf i}$ & $\vec{\symbf 0}$ & $\vec{\symbf k}$ & $-\vec{\symbf j}$\\
                \hline
                &&&\\[-1em]
                $\vec{\symbf j}$ & $-\vec{\symbf k}$& $\vec{\symbf 0}$ & $\vec{\symbf i}$\\
                \hline
                &&&\\[-1em]
                $\vec{\symbf k}$ & $\vec{\symbf j}$ & $-\vec{\symbf i}$& $\vec{\symbf 0}$
            \end{tabular}
        \end{center}

        Определённое таким образом умножение соответствует векторному умножению геометрических векторов.
    \end{example}
    \begin{definition}[Идеал]\label{def:Ideal}
        Пусть $(R,+\cdot)$~--- кольцо. Рассмотрим множество $I\subset R$, удовлетворяющее следующим условиям:
        \begin{enumerate}
            \item $\forall a\in R, b\in I \implies ab\in I$, иначе говоря $RI=I$.
            \item $(I,+)$~--- подгруппа $R$.
        \end{enumerate}

        В таком случае $I$ является подкольцом и называется \textbf{левым идеалом} кольца $R$ (потому что элемент из $I$ умножается в первом условии на элемент из $R$ слева). Аналогично, заменяя $ab$ на $ba$ в первом условии, получаем определение \textbf{правого идеала}. В коммутативных кольцах левый идеал является также и правым и называется просто \textbf{идеалом}.
    \end{definition}
    \begin{example}
        Рассмотрим кольцо целых и подкольцо чётных чисел. Очевидно, что умножение любого числа на чётное даёт чётное число, поэтому подкольцо чётных чисел является идеалом кольца целых чисел.
    \end{example}
    \begin{example}
        Рассмотрим кольцо целых чисел и подкольцо чисел, кратных $4$. Очевидно второе является идеалом первого, однако, также является идеалом идеала из предыдущего примера.
    \end{example}
    В связи с приведёнными выше примерами возникает вопрос вложенности одних идеалов в другие и существования такого понятия, как максимальный идеал.
    \begin{definition}[Максимальный идеал]\label{def:Maximal ideal}
        Идеал $I\subset R$~--- \textbf{максимальный}~$\iffdef$ $I\ne R$ и для любого идеала $\tilde I\subset R$ справедливо $I\nsubset \tilde I$
    \end{definition}
    \begin{remark}
        Максимальный идеал не единственен. Например, рассмотрим финитные последовательности $R=\left\{(a_0,a_1,\dots,a_n,0\dots)\mid n\in\NN, a_i\in\RR\right\}$ с поэлементным сложением и умножением. Рассмотрим подмножество, где $k$-ый элемент последовательности всегда равен нулю: $I_k=\{(a_0,a_1,\dots,a_n,\dots)\mid n\in\NN,a_i\in\RR,a_k=0\}$. Такое подмножество будет максимальным идеалом. Фиксируя различные $k$ будем получать различные максимальные идеалы
    \end{remark}
    Перечислим некоторые \textbf{свойства максимальных идеалов} в кольцах с единицей:
    \begin{enumerate}
        \item По Лемме Цорна \textit{в любом кольце $R$ существует максимальный идеал} и для любого идеала $I\subset R$ существует максимальный идеал $R\supset I_{\max}\supseteq I$.
        \item Все элементы, кратные некоторому необратимому образуют идеал, следовательно, по предыдущему свойству \textit{каждый необратимый элемент  содержится в некотором максимальном идеале}.

        Наоборот, если элемент обратим, то всякий идеал, его содержащий, совпадает со всем кольцом, следовательно, \textit{обратимые элементы не содержатся ни в каком максимальном идеале}.
        \item Из предыдущего свойства, \textit{все необратимые элементы кольца образуют идеал тогда и только тогда, когда этот идеал единственный максимальный.} (Также такие кольца называются \textbf{локальными}).
        \item Критерий максимальности идеала: \textit{$I\subset R$~--- максимальный идеал~$\iff$ $\factor{R}{I}$~--- поле.}
    \end{enumerate}

    Таким образом из свойства максимального идеала под номером 2 можно определить некоторые условия обратимости элемента

    \section[Нормы и нормированные алгебры]{Нормированная алгебра. Свойства нормы.}
        Определение и примеры нормированных алгебр. Свойства норм.

    \section[Банаховы алгебры. Спектр]{Банахова алгебра.\\Спектр мультипликативного функционала}
        Теорема о непустоте спектра мультипликативного функционала в банаховой алгебре.

    \section[Пространство максимальных идеалов]{Пространство максимальных идеалов. Преобразование Гельфанда.\\Теорема Тихонова.}
        Топология на пространстве максимальных идеалов.
        \begin{theorem}[Тихонова]\label{th:Tikhonov}
            content...
        \end{theorem}
        Преобразование Гельфанда для коммутативных алгебр.

    \section[Теорема Винера.]{Теорема Винера об абсолютной сходимости.}
        \begin{theorem}[Винера]\label{th:Wiener}
            об абсолютно сходящихся рядах
            \href{https://londmathsoc.onlinelibrary.wiley.com/doi/abs/10.1112/jlms/49.3.493}{something to read}
        \end{theorem}

    \section{Алгебра с инволюцией. \(\symbf{C^*}\)-алгебра.}
        \begin{theorem}[Банаха~--- Алаоглу]\label{th:Alaoglu}
            content...
        \end{theorem}
        \begin{theorem}[Стоуна~--- Вейерштрасса]\label{th:StoneWrstrs}
            content...
        \end{theorem}

    \section[Примеры пространств максимальных идеалов]{Примеры пространств максимальных идеалов.}
        \subsection{Почти периодические функции}
        \subsection{Квазипериодические функции}
    \newpage
    \begin{thebibliography}{10}
        \addcontentsline{toc}{section}{Список литературы}
        \bibitem{helemski}
        А.Я.~Хелемский. Банаховы алгебры. Ориентировочный мини-курс лекций для аспирантов
    \end{thebibliography}
\end{document}