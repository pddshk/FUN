% !TeX spellcheck = ru_RU
% !TeX TXS-program:compile = txs:///xelatex/[--shell-escape]

\documentclass[12pt]{extarticle}
\usepackage[no-math]{fontspec}
\usepackage[english,russian]{babel}
\usepackage{indentfirst}
\usepackage{xcolor}
\usepackage{tikz-cd}
\usepackage{amsmath}
\usepackage{amssymb}
\usepackage{xpatch}
\usepackage{hyperref}
\usepackage{amsthm}
\usepackage{mathrsfs}
\usepackage{yfonts}
\usepackage[inline]{enumitem}
\usepackage{unicode-math-xetex}
\usepackage{geometry}[a4papre, top=1cm, bottom=1cm, right=1cm, left=1cm]

\usepackage{xunicode}
\usetikzlibrary{babel}

% Font sepcifications
\setmainfont{palatinolinotype}[
    Extension = .ttf,
    UprightFont = *-roman,
    BoldFont = *-bold,
    ItalicFont = *-italic,
    BoldItalicFont = *-bolditalic
]

\setmathfont{latinmodern-math.otf}
\setmathfont[range={\varnothing,\setminus}]{XITS Math.otf}
\setmathfont[range=bb]{texgyrepagella-math.otf}
%%%%%%%%%%%%%%%%%%%%%%%%%%%%%%%%%%%%%%%%%%%%%%%%%%%%%%%%%%%

%\renewcommand{\baselinestretch}{1.2857142857142858094476878250134177505970001220703125}\large\rm

\makeatletter

\newcommand{\inlineitem}[1][]{%
    \ifnum\enit@type=\tw@
    {\descriptionlabel{#1}}
    \hspace{\labelsep}%
    \else
    \ifnum\enit@type=\z@
    \refstepcounter{\@listctr}\fi
    \quad\@itemlabel\hspace{\labelsep}%
    \fi
}

\xpatchcmd{\proof}{\@addpunct{.}}{}{}{}

\makeatother

\theoremstyle{definition}
\newtheorem{theorem}{\indent Теорема}[section]
\newtheorem{corollary}{\indent Следствие}%[section]
\newtheorem{lemma}{\indent Лемма}[section]
\newtheorem*{statement}{\indent Утверждение}
\newtheorem{definition}{\indent Определение}[section]
\newtheorem{example}{\indent Пример}%[section]
\newtheorem*{remark}{\indent Замечание}
\newtheorem*{question}{\indent Вопрос}
\newtheorem*{excercise}{\indent Упражнение}
\newtheorem*{corollary*}{\indent Следствие}

\renewcommand{\qedsymbol}{$\lhd$}

\addto\captionsrussian{\renewcommand{\proofname}{$\rhd$}}

%%%%%%%%%%%%%%%%%%%%%%%%%%%%%%%%%%%%%%%%%%%%%%%%%%%%%%%%%%%%%%%%%
\newcommand{\pref}[1]{(\ref{#1})}

\newcommand{\factor}[2]{\raisebox{.3em}{$#1$}\kern-.3em\Big/\kern-.3em\raisebox{-.3em}{$#2$}}


\newcommand{\brac}[1]{\left(#1\right)}
\newcommand{\fbrac}[1]{\left\{#1\right\}}
\newcommand{\sbrac}[1]{\left[#1\right]}
\newcommand{\abrac}[1]{\langle#1\rangle}

\newcommand{\eps}{\varepsilon}

\newcommand{\RR}{\mathbb{R}}
\newcommand{\NN}{\mathbb{N}}
\newcommand{\ZZ}{\mathbb{Z}}
\newcommand{\bb}[1]{\mathbb{#1}}
\newcommand{\eqby}[1]{\stackrel{\mathrm{#1}}{=}}
\newcommand{\implby}[1]{\stackrel{\mathrm{#1}}{\implies}}
\newcommand{\eqdef}{\stackrel{\mathrm{def}}{=}}
\newcommand{\ddt}{\frac{d}{dt}}
\newcommand{\iffdef}{\stackrel{\mathrm{def}}{\iff}}
\newcommand{\dd}[2]{\frac{d #1}{d #2}}

\newcommand{\at}[2]{\left.#1\right\vert_{#2}}

\newcommand{\partiald}[2]{\frac{\partial #1}{\partial #2}}
\newcommand{\partialdd}[3]{\frac{\partial^2 #1}{\partial #2 \partial #3}}

\DeclareMathOperator{\Mat}{Mat}
\DeclareMathOperator{\pr}{pr}
\DeclareMathOperator{\id}{id}
\DeclareMathOperator*{\olim}{\overline{\lim}}
\DeclareMathOperator{\transposed}{^{T}}
\newcommand{\ollim}[1]{\olim\limits_{#1}}

%%%%%%%%%%%%%%%%%%%%%%%%%%%%%%%%%%%%%%%%%%%%%%%%%%%%%%%%%%%%
\begin{document}
    \begin{titlepage}
        \null
        \vfill
        \centering\Huge\bfseries Конспект лекций по функциональному анализу
        \vfill
        \null
    \end{titlepage}
    \tableofcontents
    \newpage
    \section{Алгебра, идеал, максимальный идеал. Условия обратимости.}
    Рассмотрим векторное пространство $V$ над полем $F$.
    \begin{definition}
        \textbf{Алгеброй} называется \dots
    \end{definition}
    \begin{theorem}\label{th:some theorem}
        Любое векторное пространство можно доопределить до алгебры
    \end{theorem}
    \begin{proof}
        Доказательство
    \end{proof}
    \begin{corollary}
        Всё круто
    \end{corollary}
    \begin{example}
        Удобные макросы для математического текста можно посмотреть в преамбуле:
        $$
        \factor{\RR}{\bb Z} \implby{\ref{th:some theorem}} \at{\partiald{}{x_i}\varphi(t)}{t=p} \iffdef \ddt\id=0 \quad \sigma(A) = \{\lambda\mid\nexists(A-\lambda I)^{-1}\}
        $$
    \end{example}
    \section{Определение и примеры нормированных алгебр. Свойства норм.}

    \section{Банахова алгебра. Теорема о непустоте спектра в банаховой алгебре.}

    \section{Топология на пространстве максимальных идеалов. Теорема Тихонова. Преобразование Гельфанда для коммутативных алгебр.}

    \section{Теорема Винера об абсолютно сходящихся рядах.}

    \section{Теорема Тихонова. Преобразование Гельфанда для коммутативных алгебр}

    \section{Теорема Винера об абсолютной сходимости}

    \section{Алгебра с инволюцией. $C^*$-алгебра. Теорема Банаха~--- Алаоглу. Теорема Стоуна~--- Вейерштрасса.}

    \section{Примеры пространств максимальных идеалов (почти периодические функции, квазипериодические функции).}

\end{document}