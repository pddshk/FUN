% !TeX spellcheck = ru_RU
% !TeX TXS-program:compile = txs:///xelatex/[--shell-escape]

\documentclass[12pt]{extarticle}
\usepackage[no-math]{fontspec}
\usepackage[english,russian]{babel}
\usepackage{indentfirst}
\usepackage{xcolor}
\usepackage{tikz-cd}
\usepackage{amsmath}
\usepackage{amssymb}
\usepackage{xpatch}
\usepackage{hyperref}
\usepackage{amsthm}
\usepackage{mathrsfs}
\usepackage{yfonts}
\usepackage{titlesec}
\usepackage{unicode-math}
\usepackage[a4paper, top=1cm, bottom=1cm, right=1cm, left=1cm]{geometry}

\usepackage{xunicode}
\usetikzlibrary{babel}

% Font sepcifications
\setmainfont{PTSerif}[
    Extension = .ttf,
    UprightFont = *-Regular,
    BoldFont = *-Bold,
    ItalicFont = *-Italic,
    BoldItalicFont = *-BoldItalic
]

\setmathfont[bold-style=ISO]{latinmodern-math.otf}
\setmathfont[range={\varnothing}]{Asana-Math.otf}
\setmathfont[range=bb]{texgyrepagella-math.otf}

\AtBeginDocument{
    \renewcommand{\setminus}{\mathbin{\big\backslash}}%
}

%%%%%%%%%%%%%%%%%%%%%%%%%%%%%%%%%%%%%%%%%%%%%%%%%%%%%%%%%%%

%\renewcommand{\baselinestretch}{1.2857142857142858094476878250134177505970001220703125}\large\rm

\makeatletter

\xpatchcmd{\proof}{\@addpunct{.}}{}{}{}

\makeatother

\theoremstyle{definition}
\newtheorem{theorem}{\indent Теорема}[section]
\newtheorem{corollary}{\indent Следствие}%[section]
\newtheorem{lemma}{\indent Лемма}[section]
\newtheorem*{statement}{\indent Утверждение}
\newtheorem{definition}{\indent Определение}[section]
\newtheorem{example}{\indent Пример}%[section]
\newtheorem*{remark}{\indent Замечание}
\newtheorem*{question}{\indent Вопрос}
\newtheorem*{excercise}{\indent Упражнение}
\newtheorem*{corollary*}{\indent Следствие}

\renewcommand{\qedsymbol}{$\lhd$}

\addto\captionsrussian{\renewcommand{\proofname}{$\rhd$}}

%%%%%%%%%%%%%%%%%%%%%%%%%%%%%%%%%%%%%%%%%%%%%%%%%%%%%%%%%%%%%%%%%
\newcommand{\pref}[1]{(\ref{#1})}

\newcommand{\factor}[2]{\raisebox{.35em}{$#1$}\kern-.27em\Big/\kern-.27em\raisebox{-.35em}{$#2$}}


\newcommand{\brac}[1]{\left(#1\right)}
\newcommand{\fbrac}[1]{\left\{#1\right\}}
\newcommand{\sbrac}[1]{\left[#1\right]}
\newcommand{\abrac}[1]{\langle#1\rangle}

\newcommand{\eps}{\varepsilon}

\newcommand{\RR}{\mathbb{R}}
\newcommand{\NN}{\mathbb{N}}
\newcommand{\ZZ}{\mathbb{Z}}
\newcommand{\bb}[1]{\mathbb{#1}}
\newcommand{\eqby}[1]{\stackrel{\mathrm{#1}}{=}}
\newcommand{\implby}[1]{\stackrel{\mathrm{#1}}{\implies}}
\newcommand{\eqdef}{\stackrel{\mathrm{def}}{=}}
\newcommand{\ddt}{\frac{d}{dt}}
\newcommand{\iffdef}{\stackrel{\mathrm{def}}{\iff}}
\newcommand{\dd}[2]{\frac{d #1}{d #2}}

\newcommand{\at}[2]{\left.#1\right\vert_{#2}}

\newcommand{\partiald}[2]{\frac{\partial #1}{\partial #2}}
\newcommand{\partialdd}[3]{\frac{\partial^2 #1}{\partial #2 \partial #3}}

\DeclareMathOperator{\Mat}{Mat}
\DeclareMathOperator{\pr}{pr}
\DeclareMathOperator{\id}{id}
\DeclareMathOperator*{\olim}{\overline{\lim}}
\DeclareMathOperator{\transposed}{^{T}}
\newcommand{\ollim}[1]{\olim\limits_{#1}}

%%%
\titleformat{\section}{\centering\Large\bfseries}{\S\thesection}{3mm}{}
\titlespacing{\section}{0cm}{1.5cm}{0.5cm}

%%%%%%%%%%%%%%%%%%%%%%%%%%%%%%%%%%%%%%%%%%%%%%%%%%%%%%%%%%%%
\begin{document}
    \begin{titlepage}
        \null
        \vfill
        \centering\Huge\bfseries Конспект лекций по функциональному анализу
        \vfill
        \null
    \end{titlepage}
    \setcounter{page}{2}
    \tableofcontents
    \newpage
    \section[Алгебры, идеалы, условия обратимости.]{Алгебра, идеал, максимальный идеал.\\Условия обратимости.}
    Рассмотрим векторное пространство $V$ над полем $F$.
    \begin{definition}\label{def:Algebra}
        \textbf{Алгеброй} называется \dots
    \end{definition}
    \begin{theorem}\label{th:some theorem}
        Любое векторное пространство можно доопределить до алгебры
    \end{theorem}
    \begin{proof}
        Доказательство
    \end{proof}
    \begin{corollary}
        Всё круто
    \end{corollary}
    \begin{example}
        Удобные макросы для математического текста можно посмотреть в преамбуле:
        $$\rhd
        \factor{\RR}{\bb Z} \implby{\ref{th:some theorem}} \at{\partiald{}{x_i}\varphi(t)}{t=p} \iffdef \ddt\id=0 \quad \symbf{\sigma(A)} = \{\lambda\mid\nexists(A-\lambda I)^{-1}\} \RR\setminus\ZZ \varnothing \emptyset
        $$
    \end{example}
    \section[Нормы и нормированные алгебры]{Нормированная алгебра. Свойства нормы.}
        Определение и примеры нормированных алгебр. Свойства норм.

    \section[Банаховы алгебры. Спектр]{Банахова алгебра.\\Спектр мультипликативного функционала}
        Теорема о непустоте спектра мультипликативного функционала в банаховой алгебре.

    \section[Пространство максимальных идеалов]{Пространство максимальных идеалов. Преобразование Гельфанда.\\Теорема Тихонова.}
        Топология на пространстве максимальных идеалов.
        \begin{theorem}[Тихонова]\label{th:Tikhonov}
            content...
        \end{theorem}
        Преобразование Гельфанда для коммутативных алгебр.

    \section[Теорема Винера.]{Теорема Винера об абсолютной сходимости.}
        \begin{theorem}[Винера]\label{th:Wiener}
            об абсолютно сходящихся рядах
            \href{https://londmathsoc.onlinelibrary.wiley.com/doi/abs/10.1112/jlms/49.3.493}{something to read}
        \end{theorem}

    \section{Алгебра с инволюцией. $\symbf{C^*}$-алгебра.}
        \begin{theorem}[Банаха~--- Алаоглу]\label{th:Alaoglu}
            content...
        \end{theorem}
        \begin{theorem}[Стоуна~--- Вейерштрасса]\label{th:StoneWrstrs}
            content...
        \end{theorem}

    \section[Примеры пространств максимальных идеалов]{Примеры пространств максимальных идеалов.}
        \subsection{Почти периодические функции}
        \subsection{Квазипериодические функции}
    \newpage
    \begin{thebibliography}{10}
        \addcontentsline{toc}{section}{Список литературы}
        \bibitem{helemski}
        А.Я.~Хелемский. Банаховы алгебры. Ориентировочный мини-курс лекций для аспирантов
    \end{thebibliography}
\end{document}