% !TeX spellcheck = ru_RU
% !TeX TXS-program:compile = txs:///xelatex/[--shell-escape]

\documentclass[12pt]{extarticle}
\usepackage[no-math]{fontspec}
\usepackage[english,russian]{babel}
\usepackage{indentfirst}
\usepackage{xcolor}
\usepackage{tikz-cd}
\usepackage{amsmath}
\usepackage{amssymb}
\usepackage{xpatch}
\usepackage{hyperref}
\usepackage{amsthm}
\usepackage{mathrsfs}
\usepackage{yfonts}
\usepackage{titlesec}
\usepackage{unicode-math}
\usepackage{mathtools}
\usepackage[a4paper, top=1.5cm, bottom=1.5cm, right=1.5cm, left=1.5cm]{geometry}

\usepackage{xunicode}
\usetikzlibrary{babel}

% Font sepcifications
\setmainfont{PTSerif}[
    Extension = .ttf,
    UprightFont = *-Regular,
    BoldFont = *-Bold,
    ItalicFont = *-Italic,
    BoldItalicFont = *-BoldItalic
]

\setmathfont[bold-style=ISO]{latinmodern-math.otf}
\setmathfont[range={\varnothing}]{Asana-Math.otf}
\setmathfont[range=bb]{texgyrepagella-math.otf}

\AtBeginDocument{
    \renewcommand{\setminus}{\mathbin{\big\backslash}}%
}

%%%%%%%%%%%%%%%%%%%%%%%%%%%%%%%%%%%%%%%%%%%%%%%%%%%%%%%%%%%

%\renewcommand{\baselinestretch}{1.2857142857142858094476878250134177505970001220703125}\large\rm

\makeatletter

\xpatchcmd{\proof}{\@addpunct{.}}{}{}{}

\makeatother

\theoremstyle{definition}
\newtheorem{theorem}{\indent Теорема}[section]
\newtheorem{corollary}{\indent Следствие}%[section]
\newtheorem{lemma}{\indent Лемма}[section]
\newtheorem*{statement}{\indent Утверждение}
\newtheorem{definition}{\indent Определение}[section]
\newtheorem*{example}{\indent Пример}%[section]
\newtheorem*{remark}{\indent Замечание}
\newtheorem*{question}{\indent Вопрос}
\newtheorem*{excercise}{\indent Упражнение}
\newtheorem*{corollary*}{\indent Следствие}

\renewcommand{\qedsymbol}{$\lhd$}

\addto\captionsrussian{\renewcommand{\proofname}{$\rhd$}}

%%%%%%%%%%%%%%%%%%%%%%%%%%%%%%%%%%%%%%%%%%%%%%%%%%%%%%%%%%%%%%%%%
\newcommand{\pref}[1]{(\ref{#1})}

\newcommand{\factor}[2]{\raisebox{.35em}{$#1$}\kern-.27em\Big/\kern-.27em\raisebox{-.35em}{$#2$}}


\newcommand{\brac}[1]{\left(#1\right)}
\newcommand{\fbrac}[1]{\left\{#1\right\}}
\newcommand{\sbrac}[1]{\left[#1\right]}
\newcommand{\abrac}[1]{\langle#1\rangle}

\newcommand{\eps}{\varepsilon}

\newcommand{\RR}{\mathbb{R}}
\newcommand{\NN}{\mathbb{N}}
\newcommand{\ZZ}{\mathbb{Z}}
\newcommand{\bb}[1]{\mathbb{#1}}
\newcommand{\eqby}[1]{\stackrel{\mathrm{#1}}{=}}
\newcommand{\implby}[1]{\stackrel{\mathrm{#1}}{\implies}}
\newcommand{\eqdef}{\stackrel{\mathrm{def}}{=}}
\newcommand{\ddt}{\frac{d}{dt}}
\newcommand{\iffdef}{\stackrel{\mathrm{def}}{\iff}}
\newcommand{\dd}[2]{\frac{d #1}{d #2}}

\newcommand{\at}[2]{\left.#1\right\vert_{#2}}

\newcommand{\partiald}[2]{\frac{\partial #1}{\partial #2}}
\newcommand{\partialdd}[3]{\frac{\partial^2 #1}{\partial #2 \partial #3}}

\DeclareMathOperator{\Mat}{Mat}
\DeclareMathOperator{\pr}{pr}
\DeclareMathOperator{\id}{id}
\DeclareMathOperator*{\olim}{\overline{\lim}}
\DeclareMathOperator{\transposed}{^{T}}
\newcommand{\ollim}[1]{\olim\limits_{#1}}

%%%
\titleformat{\section}{\centering\Large\bfseries}{\S\thesection}{3mm}{}
\titlespacing{\section}{0cm}{1.5cm}{0.5cm}

%%%%%%%%%%%%%%%%%%%%%%%%%%%%%%%%%%%%%%%%%%%%%%%%%%%%%%%%%%%%
\begin{document}
    \begin{titlepage}
        \null
        \vfill
        \centering\Huge\bfseries Конспект лекций\\по функциональному анализу
        \vfill
        \null
    \end{titlepage}
    \setcounter{page}{2}
    \tableofcontents
    \newpage
    \section[Алгебры, идеалы, условия обратимости.]{Алгебра, идеал, максимальный идеал.\\Условия обратимости.}
    Рассмотрим векторное пространство $A$ над полем $F$ (или, вообще говоря, модуль $A$ над кольцом $R$).
    \begin{definition}[Алгебра]\label{def:Algebra}
        Пусть $R$~--- коммутативное кольцо с единицей, $A$~--- модуль над этим кольцом ($R$-модуль). Снабдим $A$ билинейным ассоциативным отображением $f\colon A\times A\to A$ и назовём это отображение умножением. Тогда $(A, +, \cdot)$ является кольцом.

        Полученная структура называется \textbf{алгеброй над кольцом $\symbf R$}.
    \end{definition}
    \begin{remark}
        В определении алгебры требовалась ассоциативность умножения. Вообще говоря это необязательно, однако в данном конспекте будут рассматриваться только ассоциативные алгебры, поэтому для краткости будем считать умножение ассоциативным.

        Также далее будут рассматриваться в основном векторные пространства над полями (частный случай модуля над кольцом).
    \end{remark}
    \begin{theorem}\label{th:Constructing algebra}
        Любое векторное пространство $V$ над полем $F$ можно доопределить до алгебры
    \end{theorem}
    \begin{proof}
        Доказательство проведём для конечномерного случая. Для бесконечномерного аналогично.

        Выберем некоторый базис векторного пространства $\{e_1,e_2,\dots,e_n\}$ и произвольные $n^2$ векторов $v_{11}, v_{12}, \dots, v_{nn}$. Определим умножение следующим образом:
        \begin{gather*}
            e_i\cdot e_j = v_{ij}\\
            \forall \alpha,\beta\in F, u,v\in V\\
            \alpha(\beta u) = (\alpha\beta)u\\
            \alpha(u + v)=\alpha u + \alpha v\\
            u(\alpha+\beta)=u\alpha+u\beta
        \end{gather*}

        Тогда в силу определения базиса имеем для любых векторов $a,b\in V$
        \begin{gather*}
            a=a_1e_1+a_2e_2+\dots+a_ne_n\\
            b=b_1e_1+b_2e_2+\dots+b_ne_n\\
            a\cdot b= (a_1e_1+a_2e_2+\dots+a_ne_n)\cdot(b_1e_1+b_2e_2+\dots+b_ne_n)=\\
            =a_1b_1v_{11}+a_1b_2v_{12}+\dots+a_nb_nv_{nn}
        \end{gather*}
    \end{proof}
    \begin{corollary*}
        Все дополнительные свойства (коммутативность, антикоммутативность и т.д.), которыми умножение обладает на базисных векторах, продолжаются на всё пространство.
    \end{corollary*}
    \begin{example}
        $V = \RR^3$, базис $\symbf{\vec i,\vec j,\vec k}$, умножение базисных векторов:
        \begin{center}
            \begin{tabular}{c|c|c|c}
                $\times$ & $\vec{\symbf i}$ & $\vec{\symbf j}$ & $\vec{\symbf k}$\\
                \hline
                &&&\\[-1em]
                $\vec{\symbf i}$ & $\vec{\symbf 0}$ & $\vec{\symbf k}$ & $-\vec{\symbf j}$\\
                \hline
                &&&\\[-1em]
                $\vec{\symbf j}$ & $-\vec{\symbf k}$& $\vec{\symbf 0}$ & $\vec{\symbf i}$\\
                \hline
                &&&\\[-1em]
                $\vec{\symbf k}$ & $\vec{\symbf j}$ & $-\vec{\symbf i}$& $\vec{\symbf 0}$
            \end{tabular}
        \end{center}

        Определённое таким образом умножение соответствует векторному умножению геометрических векторов.
    \end{example}
    \begin{definition}[Идеал]\label{def:Ideal}
        Пусть $(R,+\cdot)$~--- кольцо. Рассмотрим множество $I\subset R$, удовлетворяющее следующим условиям:
        \begin{enumerate}
            \item $\forall a\in R, b\in I \implies ab\in I$, иначе говоря $RI=I$.
            \item $(I,+)$~--- подгруппа $R$.
        \end{enumerate}

        В таком случае $I$ является подкольцом и называется \textbf{левым идеалом} кольца $R$ (потому что элемент из $I$ умножается в первом условии на элемент из $R$ слева). Аналогично, заменяя $ab$ на $ba$ в первом условии, получаем определение \textbf{правого идеала}. В коммутативных кольцах левый идеал является также и правым и называется просто \textbf{идеалом}.
    \end{definition}
    \begin{example}
        Рассмотрим кольцо целых и подкольцо чётных чисел. Очевидно, что умножение любого числа на чётное даёт чётное число, поэтому подкольцо чётных чисел является идеалом кольца целых чисел.
    \end{example}
    \begin{example}
        Рассмотрим кольцо целых чисел и подкольцо чисел, кратных $4$. Очевидно второе является идеалом первого, однако, также является идеалом идеала из предыдущего примера.
    \end{example}
    В связи с приведёнными выше примерами возникает вопрос вложенности одних идеалов в другие и существования такого понятия, как максимальный идеал.
    \begin{definition}[Максимальный идеал]\label{def:Maximal ideal}
        Идеал $I\subset R$~--- \textbf{максимальный}~$\iffdef$ $I\ne R$ и для любого идеала $\tilde I\subset R$ справедливо $I\nsubset \tilde I$
    \end{definition}
    \begin{remark}
        Максимальный идеал не единственен. Например, рассмотрим финитные последовательности $R=\left\{(a_0,a_1,\dots,a_n,0\dots)\mid n\in\NN, a_i\in\RR\right\}$ с поэлементным сложением и умножением. Рассмотрим подмножество, где $k$-ый элемент последовательности всегда равен нулю: $I_k=\{(a_0,a_1,\dots,a_n,\dots)\mid n\in\NN,a_i\in\RR,a_k=0\}$. Такое подмножество будет максимальным идеалом. Фиксируя различные $k$ будем получать различные максимальные идеалы
    \end{remark}
    Перечислим некоторые \textbf{свойства максимальных идеалов} в кольцах с единицей:
    \begin{enumerate}
        \item По Лемме Цорна \textit{в любом кольце $R$ существует максимальный идеал} и для любого идеала $I\subset R$ существует максимальный идеал $R\supset I_{\max}\supseteq I$.
        \item Все элементы, кратные некоторому необратимому образуют идеал, следовательно, по предыдущему свойству \textit{каждый необратимый элемент  содержится в некотором максимальном идеале}.

        Наоборот, если элемент обратим, то всякий идеал, его содержащий, совпадает со всем кольцом, следовательно, \textit{обратимые элементы не содержатся ни в каком максимальном идеале}.
        \item Из предыдущего свойства, \textit{все необратимые элементы кольца образуют идеал тогда и только тогда, когда этот идеал единственный максимальный.} (Также такие кольца называются \textbf{локальными}).
        \item Критерий максимальности идеала: \textit{$I\subset R$~--- максимальный идеал~$\iff$ $\factor{R}{I}$~--- поле.}
    \end{enumerate}

    Таким образом из свойства максимального идеала под номером 2 можно определить некоторые условия обратимости элемента

    \section[Нормы и нормированные алгебры]{Нормированная алгебра. Свойства нормы.}
        Определение и примеры нормированных алгебр. Свойства норм.

    \section[Банаховы алгебры. Спектр]{Банахова алгебра.\\Спектр мультипликативного функционала}
        Теорема о непустоте спектра мультипликативного функционала в банаховой алгебре.

    \section[Пространство максимальных идеалов]{Пространство максимальных идеалов. Преобразование Гельфанда.\\Теорема Тихонова.}
        Топология на пространстве максимальных идеалов.
        \begin{theorem}[Тихонова]\label{th:Tikhonov}
            content...
        \end{theorem}
        Преобразование Гельфанда для коммутативных алгебр.

    \section[Теорема Винера.]{Теорема Винера об абсолютной сходимости.}
    В данном параграфе 2 пункта:
	\begin{itemize}
		\item \textbf{Анализ литературы} — о том, где можно прочитать про эту теорему, и в каких научных статьях она используется
		\item \textbf{Необходимые сведения и теорема Винера} — изложение, соответствующее нашему спецкурсу по банаховым алгебрам
	\end{itemize}
    \subsection{Анализ литературы}
    Рассмотрим источники, в которых присутствует теорема Винера. Для начала покажем, как она выглядит там, где её можно найти:
    	\begin{enumerate}
			\item \textbf{Хелемский А. Я.} Банаховы алгебры. Ориентировочный мини-курс лекций для аспирантов, 2015\\
			\textit{Страница 13}\\
			...\\
			Рассмотрим объект классического гармонического анализа: $2\pi$-периодическую функцию $f:\mathbb{R}\rightarrow\mathbb{C}$, разлагающуюся в абсолютно сходящийся ряд Фурье, то есть представимую в виде $f(t)=\sum_{n=-\infty}^\infty c_n e^{int}; \sum_{n=-\infty}^\infty |c_n|<\infty$. Множество таких функций обозначим через $W$.
			\textbf{Теорема Винера}
				Если функция $f\in W$ нигде не обращается в нуль, то функция $1/f$ также лежит в $W$.
				
				Первоначальное доказательство Винера нетривиально и основано — по крайней мере, по мнению вашего лектора на изощренной аналитической технике. А вот как доказывает эту теорему Гельфанд:
			\begin{proof}
				Как легко проверить, $W$ является коммутативной банаховой алгеброй относительно поточечных операций и нормы $||f||:=\sum_{n=-\infty}^\infty |c_n|$, где $c_n$;$n\in\mathbb{Z}$ — коэффициенты Фурье нашей функции. Найдём её гельфандовский спектр. 
				
				Очевидно, что каждое число $s\in[0,2\pi)$ доставляет ненулевой <<характер означивания>> $\chi_s:f\mapsto f(s)$. Покажем, что других ненулевых характеров у $W$ нет.
				
				Возьмём произвольный $\chi\in\Omega$. Пусть $\chi(e^{it})=\lambda$. Тогда, конечно, $\chi(e^{-it})=1/\lambda$, а значит, $\chi(e^{int})=\lambda^n$ для всех $n\in\mathbb{Z}$. Но так как $||e^{it}||=||e^{-it}||=1$, то в силу Предложения 1 выполнено $|\lambda|,|1/\lambda|\leq1$. Отсюда $|\lambda|=1$, и $\lambda=e^{is}$ для некоторого $s\in[0,2\pi)$. Но тогда для любой $f\in W$, разлагающейся в ряд $\sum_{n=-\infty}^\infty c_n e^{int}$, абсолютная сходимость этого ряда влечёт 
				$$\chi(f)=\sum_{n=-\infty}^\infty c_n \chi(e^{ins})=f(s)$$
				Таким образом, в терминах преобразования Гельфанда $\hat{f}(\chi)=f(s)$. Поэтому, раз $f$ нигде не равна нулю, то же верно и для функции $\hat{f}$ на $\Omega$. Но тогда, в силу утверждения $(ii)$ теоремы Гельфанда, $f$ — обратимый элемент алгебры $W$. Это как раз то, что нам нужно.	
			\end{proof}
		...
		\item \textbf{Р. Эдвардс} Ряды Фурье в современном изложении в 2-х томах, Том 2, 1985\\
		\textit{Страница 44}\\
		...\\
		Из теоремы 11.4.10 мы можем теперь получить теорему Винера [Wi, стр. 121]: если функция $f\in A$ удовлетворяет условию $f(x)\neq 0$ для всех вещественных $x$, то $1/f\in A$. 
		
		Аналогичным образом из 11.4.15 можно получить обобщение Леви теоремы Винера, а именно: если $f\in A$ и если функция $\Phi$ определена и аналитична на некотором открытом множестве, содержащем $f(T)$, то $\Phi\circ f\in A$.
		
		...
	\end{enumerate}
  	 Ясно, что приведённые куски текста имеют иллюстративную роль, так как они вырваны из контекста. Их цель — помочь с выбором источника, в котором можно найти соответствующие пояснения. Ниже рассмотрим все статьи, где затрагивается теорема Винера. Русскоязычные статьи на mathnet.ru:
  	\begin{enumerate}
  		\item \textbf{В. Е. Струков, И. И. Струкова} О теореме Винера в исследовании периодических на бесконечности функций относительно подпространств исчезающих на бесконечности функций. \textit{ТВИМ, 2019, № 4, 78–91}
		\item \textbf{А. В. Загороднюк, М. А. Митрофанов} Аналог теоремы Винера для бесконечномерных банаховых пространств. \textit{Математические заметки, Том 97, выпуск 2 февраль 2015}
		\item \textbf{И. И. Струкова} О теореме Винера для периодических на бесконечности функций. \textit{Сибирский математический журнал. Январь-февраль, 2016. Том 57, №1}
		\item \textbf{В. Е. Струков, И. И. Струкова} Теорема Винера в исследовании почти периодических на бесконечности функций. \textit{Прикладная математика \& Физика, 2019, том 51, выпуск 3, страницы 387–401}
		\item \textbf{И. И. Струкова} Теорема Винера для периодических на бесконечности функций с рядами Фурье, суммируемыми с весом. \textit{Уфимский математический журнал, 2013, том 5, выпуск 3, страницы 144–152}
		\item \textbf{И. И. Струкова} Теорема Винера для периодических на бесконечности функций \textit{Известия Саратовского университета. Новая серия. Серия: Математика. Механика. Информатика, 2012, том 12, выпуск 4, страницы 34–41}
		\item \textbf{А. Г. Баскаков} Теорема Винера и асимптотические оценки элементов обратных матриц \textit{Функц.
			анализ и его прил., 1990, том 24, выпуск 3, 64–65}
		\item \textbf{Б. А. Рогозин} Асимптотика коэффициентов в теоремах Леви–Винера об абсолютно сходящихся тригонометрических рядах \textit{Сибирский математический журнал, 1973, том 14, номер 6, страницы 1304–1312}\\
		 К этой статье имеется замечание:\\ \textbf{Л. Марки} Замечание к статье Б. А. Рогозина “Асимптотика коэффициентов в теоремах Леви–Винера об абсолютно сходящихся тригонометрических рядах” \textit{Сибирский математический журнал, 1977, том 18, номер 4, страницы 944–946}
		 \item \textbf{Г. Н. Агаев} Теорема типа Винера для рядов по функциям Уолша. \textit{Докл. АН СССР, 142:4 (1962), 751–753}
		 \item \textbf{С. С. Волосивец} О весовых аналогах теорем Винера и Леви для рядов Фурье-Виленкина. \textit{Саратовский государственный университет, кафедра теории функций и приближений, 2011}
		 \item \textbf{Е. М. Дынькин} Индивидуальные теоремы типа Винера–Леви для рядов и интегралов Фурье. \textit{Зап. научн. сем. ЛОМИ, 1971, том 22, 181–182}
  	\end{enumerate}
  	Англоязычные статьи на arxiv.org:
  	\begin{enumerate}
		\item \textbf{PRAKASH A. DABHI AND KARISHMAN B. SOLANKI} VECTOR VALUED BEURLING ALGEBRA ANALOGUES OF
		WIENER’S THEOREM \textit{arXiv:2210.04444 [math.FA], 2022}
		\item \textbf{GRUIA ARSU} ON KATO-SOBOLEV SPACES. THE WIENER-LEVY THEOREM FOR KATO-SOBOLEV ALGEBRAS $H^s_{ul}$ \textit{arXiv:1010.0815 [math.FA], 2010}
		\item \textbf{S. J. Bhatt, H. V. Dedania} Beurling algebra analogues of the classical theorems of Wiener and Levy on absolutely convergent Fourier series \textit{	arXiv:math/0310291 [math.CV], 2003}
  	\end{enumerate}
  	Тщательно проанализировав вышеуказанные и иные источники, имеем следующие \textbf{выводы}:
  		\begin{itemize}
			\item рассматриваемая теорема Винера встречается либо как \textbf{теорема Винера}, либо как \textbf{теорема Винера-Леви} (согласно обобщению Леви)
			\item всего имеется 14 вышеперечисленных статей
			\item на порядок чаще на ресурсах mathnet.ru и arxiv.org встречается \textbf{теорема Винера-Пэли} (она же \textbf{теорема Пэли-Винера-Гельфанда}, она же \textbf{тауберова теорема Винера}) (100+ статей)
			\item согласно количеству статей, равному 14, и датам их выхода (2022 — последний год), можно сделать \textbf{основной вывод}: теорема Винера представляет научный интерес.
  		\end{itemize}
  	\textbf{Замечание}
  	
  	Стоит отметить, что, возможно, некоторое количество статей было упущено. Это связано с тем, что искались только те статьи, в которых каким-либо образом присутствует "Винер"\ ("Wiener"). Кроме того, у любой теоремы есть обобщение. Для теоремы Винера обобщением является \textbf{теорема Бохнера-Филлипса}. О ней можно прочитать в следующей статье: \textbf{В. Е. Струков} О СТРУКТУРЕ ОПЕРАТОРА, ОБРАТНОГО К ИНТЕГРАЛЬНОМУ ОПЕРАТОРУ
  	СПЕЦИАЛЬНОГО ВИДА \textit{Изв. Сарат. ун-та. Нов. сер. Сер. Математика. Механика. Информатика. 2013. Т. 13, вып. 2, ч. 1}. Там же упоминается и теорема Винера, однако данная статья — про обобщение, насчёт которого имеется ещё ряд статей, поэтому в список выше она не включена: в противном случае пришлось бы включать статьи про обобщения обобщений и так далее.
  	\subsection{Необходимые сведения и теорема Винера}
  	\noindent
  		Рассмотрим ряды:
  		$$\frac{1}{2}a_0+\sum_{n=1}^{\infty} a_n \cos nx+b_n \sin nx$$ $$a_n,b_n\in\mathbb{R}$$
  		и
  		$$\sum_{n=-\infty}^\infty c_n e^{inx}$$
  		$$c_n\in\mathbb{C}$$
  		Если существует такая функция $f\in L([-\pi,\pi])$, что $$f(x)=\frac{1}{2}a_0+\sum_{n=1}^{\infty} a_n \cos nx+b_n \sin nx=\sum_{n=-\infty}^\infty c_n e^{inx}$$
  		то соответствующие ряды называются \textbf{рядами Фурье функции} $f$. При этом $a_n,b_n, c_n$ называются \textbf{коэффициентами Фурье функции} $f$. Ряд Фурье называется \textbf{абсолютно сходящимся}, если $\sum_{n=-\infty}^\infty |c_n|<\infty$. Итак,
        \begin{theorem}[Винер]\label{th:Wiener}
        $A$ — множество функций из $\mathbb{R}$ в $\mathbb{C}$, разлагающихся в абсолютно сходящийся ряд Фурье. Тогда
      $$
    \begin{drcases}
		f\in A\\
		 \forall x\in\mathbb{R}: f(x)\neq 0
    \end{drcases}\implies 1/f\in A
    $$
    где $1/f$ — функция, обратная к $f$.
    \begin{proof}
		Покажем, что $A$ — коммутативная банахова алгебра. Для этого нам необходимо:
		\begin{enumerate}
			\item Задать 3 операции (сложение, умножение, умножение на скаляр) так, чтобы получить структуру алгебры над некоторым полем, т.е. векторного пространства
			\item Задать норму так, чтобы векторное пространство было полным по метрике, порождённым нормой (для банахова пространства). При этом норма должна быть согласована с произведением (для банаховой алгебры):
			$$\forall f,g\in A: ||fg||\leq ||f||\cdot||g||$$
		\end{enumerate}
		\begin{enumerate}
			\item Поточечно зададим сложение, умножение и умножение на скаляр:
				$$
			f+g:=u: x\mapsto f(x)+g(x)
			$$
			$$fg:=v:x\mapsto f(x)g(x)$$
			$$fc:=f(x)c,c\in\mathbb{C}$$
			откуда из того что $\mathbb{C}$ — алгебра над собой, т.е. $\mathbb{C}$-алгебра, следует, что $A$ — $\mathbb{C}$-алгебра.
			\item Зададим норму $||f||:=\sum_{n=-\infty}^\infty |c_n|$ — сумма модулей коэффициентов Фурье. Из теории следует, что $A$ — банахова алгебра.
		\end{enumerate}
		Рассмотрим характер $\chi$ — элемент спектра. Пусть $\chi(e^{it})=\lambda\in\mathbb{C}$. Так как характер является, в частности, гомоморфизмом алгебр, то $\chi(e^{-it})=1/\lambda$ и $\chi(e^{nit})=\lambda^n,n\in\mathbb{Z}$. Характер банаховой алгебры является ограниченным и даже сжимающим функционалом, т.е. $||\chi||\leq 1$, так что $|\lambda|, |1/\lambda|\leq 1$. Если $|\lambda|<1$, то $|1/\lambda|>1$ — противоречие. Поэтому $|\lambda|=1$, т.е. $\lambda$ лежит на комплексной окружности радиуса $1$; это значит, что $\exists s\in [0,2\pi):\lambda=e^{is}$. Тогда если $f\in A$ раскладывается в ряд $f(t)=\sum_{n=-\infty}^{\infty} c_n e^{int}$, то
		$$\chi(f)=\sum_{n=-\infty}^{\infty} c_n \chi (e^{ins})=f(s)$$
		Так как элемент банаховой алгебры обратим тогда и только тогда, когда любой характер элемента ненулевой (теорема Гельфанда), то в силу произвольности выбора $\chi$ и условия теоремы $\forall x\in\mathbb{R}: f(x)\neq 0$ следует, что $1/f\in A$.
    \end{proof}
        \end{theorem}

    \section{Алгебра с инволюцией. \(\symbf{C^*}\)-алгебра.}
        \begin{theorem}[Банаха~--- Алаоглу]\label{th:Alaoglu}
            content...
        \end{theorem}
        \begin{theorem}[Стоуна~--- Вейерштрасса]\label{th:StoneWrstrs}
            content...
        \end{theorem}

    \section[Примеры пространств максимальных идеалов]{Примеры пространств максимальных идеалов.}
        \subsection{Почти периодические функции}
        \subsection{Квазипериодические функции}
    \newpage
    \begin{thebibliography}{10}
        \addcontentsline{toc}{section}{Список литературы}
        \bibitem{helemski}
        А.Я.~Хелемский. Банаховы алгебры. Ориентировочный мини-курс лекций для аспирантов
    \end{thebibliography}
\end{document}
